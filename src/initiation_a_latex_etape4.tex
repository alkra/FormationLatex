\documentclass[10pt,a4paper]{report} % Creation d'un rapport avec : 
% taille police : 10 pt
% format : document a4

% Définition des encodages utilisés
\usepackage[T1]{fontenc} % encodage utilisé pour l'impression de caractères (dans le .pdf)
\usepackage[utf8]{inputenc} % encodage utilisé pour la saisie du document (dans le .tex)
\usepackage{lmodern} % utilise la police vectorielle (utile lors d'une impression papier du document) -> c'est plus joli

% Configuration de la page de garde
\title{Initiation à Latex}
\author{Vous-même}

% Configuration de la langue utilisée (pour les mots clés comme "chapitre", pour le découpage des mots à la fin des lignes...)
\usepackage[french]{babel}

% Génération du document
\begin{document}
\maketitle
\tableofcontents

Mon premier document en Latex. Il a l'air joli, ça me plaît.
Nous allons maintenant structurer le fichier pour faire un joli rapport.

\part{Contexte}
C'est quand même important de commencer par présenter le contexte du travail.
\chapter{Introduction}
Ici nous venons de créer un nouveau chapitre.
\section{Pourquoi cette formation}
Parce que Latex va vous permettre de rédiger un document en équipe, sans se soucier des préférences d'affichage. Latex sera respectueux de l'utilisation de votre système d'exploitation. Il est utilisable que vous soyez sous windows, sous mac ou sous linux.
\subsection{Le lieu}
Durant votre scolarité à l'ENSG, vous allez rédiger bon nombre de rapports. 
\subsubsection{Vertigeo dans tout ça}
Parce que faire de la pub pour la Junior Entreprise, c'est cool.
\paragraph{Rédigé, c'est structuré}
La mise en forme est gérée par Latex, on peut donc en théorie se concentrer sur le contenu du rapport.
\subparagraph{Toujours plus}
Un niveau en dessous du paragraphe. Cela peut être utile dans certaines situations

L'intérêt de cette formation est de vous initier à Latex. Car bien qu'il permet de se concentrer sur le contenu, si son fonctionnement n'est pas compris, ou si personne ne vous a montré un exemple, vous allez passer plus de temps à gérer un problème de mise en forme plutôt qu'écrire du contenu. (Exemple d'une image)

\subsection{Une autre subsection}
Celle ci est vide. Remarquez que le numéro s'est mis à jour.
\section{Quelques conseils}
La table des matières est mise à jour après deux compilations sans modification de la structure entre temps. La première compilation permet à Latex de savoir quelles vont être les entrées. Le deuxième passage va l'afficher.
\subsection*{Structure non numérotée}
Pour ajouter une structure qui n'apparaît pas dans la légende, il suffit de rajouter * après la commande. En faisant ceci, Latex ignorera cette entrée, ce qui signifie pas de numéro, ni de place dans la table des matières. Peut être pratique pour une section Introduction ou Conclusion par exemple.

\subsection{Mettre à jour des éléments}
Le texte peut être mis en valeur par quelques commandes. Il est possible de \textbf{mettre en gras}, \emph{mettre en italique}.

\section*{Conclusion}
\addcontentsline{toc}{section}{Conclusion}
(Cette entrée apparaît dans la table des matières.)
Au final, nous avons appris quelque chose. Je connais maintenant la base de Latex. Il m'a fallu quelques heures de formation. Je peux désormais rédiger des rapports qui font pro. Je suis content.
\end{document}