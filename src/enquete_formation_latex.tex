\documentclass[a4paper,11pt,BCOR10mm,oneside,headsepline]{scrartcl}

\usepackage[T1]{fontenc}
\usepackage[french]{babel}
\usepackage[utf8]{inputenc}

\usepackage{wasysym}% provides \ocircle and \Box
\usepackage{enumitem}% easy control of topsep and leftmargin for lists
\usepackage{color}% used for background color
\usepackage{forloop}% used for \Qrating and \Qlines
\usepackage{ifthen}% used for \Qitem and \QItem
\usepackage{typearea}
\usepackage{caption}
\usepackage{subfigure}
\usepackage{tikz}

\areaset{17cm}{26cm}
\setlength{\topmargin}{-1cm}
\usepackage{scrpage2}
\pagestyle{scrheadings}
\ihead{20 octobre 2016}
\ohead{Loïc Messal}
\chead{Enquête formation Vertigéo}
\cfoot{Service formation Vertigéo -- La formation par les étudiants, pour les étudiants ! }


%% \Qq = Questionaire question. Oh, this is just too simple. It helps
%% making it easy to globally change the appearance of questions.
\newcommand{\Qq}[1]{#1}

%% \QO = Circle or box to be ticked. Used both by direct call and by
%% \Qrating and \Qlist.
\newcommand{\QO}{$\Box$}% or: $\ocircle$

%% \Qrating = Automatically create a rating scale with NUM steps, like
%% this: 0--0--0--0--0.
\newcounter{qr}
\newcommand{\Qrating}[1]{\QO\forloop{qr}{1}{\value{qr} < #1}{---\QO}}

%% \Qline = Again, this is very simple. It helps setting the line
%% thickness globally. Used both by direct call and by \Qlines.
\newcommand{\Qline}[1]{\noindent\rule{#1}{0.6pt}}

%% \Qlines = Insert NUM lines with width=\linewith. You can change the
%% \vskip value to adjust the spacing.
\newcounter{ql}
\newcommand{\Qlines}[1]{\forloop{ql}{0}{\value{ql}<#1}{\vskip0em\Qline{\linewidth}}}

%% \Qlist = This is an environment very similar to itemize but with
%% \QO in front of each list item. Useful for classical multiple
%% choice. Change leftmargin and topsep accourding to your taste.
\newenvironment{Qlist}{%
\renewcommand{\labelitemi}{\QO}
\begin{itemize}[leftmargin=1.5em,topsep=-.5em]
}{%
\end{itemize}
}

%% \Qtab = A "tabulator simulation". The first argument is the
%% distance from the left margin. The second argument is content which
%% is indented within the current row.
\newlength{\qt}
\newcommand{\Qtab}[2]{
\setlength{\qt}{\linewidth}
\addtolength{\qt}{-#1}
\hfill\parbox[t]{\qt}{\raggedright #2}
}

%% \Qitem = Item with automatic numbering. The first optional argument
%% can be used to create sub-items like 2a, 2b, 2c, ... The item
%% number is increased if the first argument is omitted or equals 'a'.
%% You will have to adjust this if you prefer a different numbering
%% scheme. Adjust topsep and leftmargin as needed.
\newcounter{itemnummer}
\newcommand{\Qitem}[2][]{% #1 optional, #2 notwendig
\ifthenelse{\equal{#1}{}}{\stepcounter{itemnummer}}{}
\ifthenelse{\equal{#1}{a}}{\stepcounter{itemnummer}}{}
\begin{enumerate}[topsep=2pt,leftmargin=2.8em]
\item[\arabic{itemnummer}#1.] #2
\end{enumerate}
}

%% \QItem = Like \Qitem but with alternating background color. This
%% might be error prone as I hard-coded some lengths (-5.25pt and
%% -3pt)! I do not yet understand why I need them.
\definecolor{bgodd}{rgb}{0.8,0.8,0.8}
\definecolor{bgeven}{rgb}{0.9,0.9,0.9}
\newcounter{itemoddeven}
\newlength{\gb}
\newcommand{\QItem}[2][]{% #1 optional, #2 notwendig
\setlength{\gb}{\linewidth}
\addtolength{\gb}{-5.25pt}
\ifthenelse{\equal{\value{itemoddeven}}{0}}{%
\noindent\colorbox{bgeven}{\hskip-3pt\begin{minipage}{\gb}\Qitem[#1]{#2}\end{minipage}}%
\stepcounter{itemoddeven}%
}{%
\noindent\colorbox{bgodd}{\hskip-3pt\begin{minipage}{\gb}\Qitem[#1]{#2}\end{minipage}}%
\setcounter{itemoddeven}{0}%
}
}

%%%%%%%%%%%%%%%%%%%%%%%%%%%%%%%%%%%%%%%%%%%%%%%%%%%%%%%%%%%%
%% End of questionnaire command definitions %%
%%%%%%%%%%%%%%%%%%%%%%%%%%%%%%%%%%%%%%%%%%%%%%%%%%%%%%%%%%%%

\begin{document}

\begin{center}
	\LARGE Formation initiation \\
	\textbf{\Huge\LaTeX}
\end{center}\vskip1em

\noindent Bienvenue à cette séance d'initiation à \LaTeX. Merci de bien vouloir remplir la première partie de ce document avant de commencer la formation et la seconde partie après l'avoir terminée.

\section*{Avant la formation}

\Qitem{\Qq{Auparavant, aviez-vous entendu parler de Latex ?} \Qtab{5.5cm}{\QO{} Oui
	\hskip0.5cm \QO{} Non}}

\Qitem{\Qq{Si oui, l'aviez-vous déjà utilisé ?} \Qtab{5.5cm}{\QO{} Oui
	\hskip0.5cm \QO{} Non}}

\Qitem{ \Qq{Qu'est-ce qui vous a poussé à vous inscrire : }
	\begin{Qlist}
		\item le mail
		\item le sujet traité (Latex)
		\item le programme détaillé
		\item le présentateur
		\item vos camarades
		\item le fait que ce soit organisé par Vertigéo
		\item parce que c'est gratuit
		\item parce que c'était plus pratique pour vous de rester afin d'assister au Ciné-Club
		\item Autre : \Qline{4cm}
	\end{Qlist}
}

\Qitem{ \Qq{Qu'attendez-vous de la formation ? } \Qlines{2} }


\section*{Après la formation}
\begin{figure}[!h]%
	\centering
	\subfigure[Avant la formation]{%
		\begin{tikzpicture}
			\draw (0,0) rectangle +(4,4);
		\end{tikzpicture}
	}%
	\qquad
	\subfigure[Après la formation]{%
		\begin{tikzpicture}
			\draw (0,0) rectangle +(4,4);
		\end{tikzpicture}
	}%
\end{figure}
\Qitem{\Qq{Dessinez votre tête dans les cadres ci-dessus.}}

\Qitem{ \Qq{Avez-vous envie d'une nouvelle formation sur un autre sujet (une technologie, un logiciel, un outil, une méthode, autre...) ?} \Qtab{5.5cm}{\QO{} Oui
	\hskip0.5cm \QO{} Non} \\
	Précisez : \Qline{10cm} }

\Qitem{ \Qq{Je suis sûr que vous avez des compétences ! Pensez-vous organiser une formation à votre tour ?} \\
	\hskip0.4cm \QO{}
	Oui \hskip0.5cm \QO{} Non }

\Qitem{ \Qq{Vos remarques sur l'expérience qui vous a été proposée :} \Qlines{2} }


\end{document}